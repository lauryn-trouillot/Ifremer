\documentclass{article}
\usepackage{graphicx} % Required for inserting images
\usepackage[backend=biber,style=numeric]{biblatex} % Use biblatex for bibliography

\addbibresource{ref.bib} % The bib file must be in the same directory

\title{Résumé des références}
\author{Lauryn}
\date{September 2024}

\begin{document}

\maketitle
\section{Références}
Article \cite{reguera2024dinophysis}:
\begin{itemize}
    \item Dinophysis cause des gastroenterites, après la consommation de coquillage contaminé par ces toxines
    \item Ces espèces sont ciblé par des programme de protection de la santé publique.
\end{itemize}
\begin{itemize}
    \item Dinophysis a fait l'objet d'un examen plus approfondi lorsque ont mis en évidence l'autofluorescence orange et l'ultrastructure de ses plastes semblables à ceux des cryptophytes.
\end{itemize}
\begin{itemize}
    \item Les blooms de dinophysis sont liés a la stratification thermohaline. 
\end{itemize}
\begin{itemize}
    \item Il est difficile de faire pousser dinophysis en laboratoire = nécessite la chaine alimentaire Dinophysis-Mesodinium-Teleaulax.
\end{itemize}
\begin{itemize}
    \item Les dinoflagellés sont des protistes ayant cycles de vie complexes et hétéromorphes avec des transitions entre des stades avec des ploïdies et des morphologies différentes. La différenciation des cellules végétatives en gamètes peut procéder à une « dépauperating division », chaque cellule mère produisant deux cellules filles avec une ploïdie et une morphologie différentes.
\end{itemize}
\begin{itemize}
    \item Les dinoflagellés sont des organismes holoplanctonique. 
\end{itemize}
\begin{itemize}
    \item Explication du cycle de vie
    \item Dinophysis toxiques : D. acuminata complex, D. acuta, D. caudata, D. fortii and D.norvegica => augmentent le nombre de blooms dans les eaux cotières. 
    \item Le Dinophysis acuminata complex fait référence à un groupe d'espèces, qui sont morphologiquement très similaires et parfois difficiles à distinguer les unes des autres
    \item Une grande majorité des dinoflagellés vivant librement sont mixoplanktoniques, c'est-à-dire qu'ils ont la capacité de combiner la phototrophie et la phagotrophie
\begin{itemize}
    \item La photosynthèse est réalisée soit à l'aide de cellules permanentes (constitutives) ou temporaires (non constitutifs) des plastes volés (kleptoplastides) d'une variété (généraliste) ou d'un groupe très sélectionné de proies (spécialiste plastidique)
\end{itemize}
\item  En plus des proies dinophysis a besoin de lumière et de nutriment pour faire la photosynthèse.
\item Les meilleures souche de Mesodinium pour la croissance de Dinophysis
\item Dinophysis a réalisé la photosynthèse avec ses propres plastes (constitutifs) ou avec ceux conservés des proies (Garcı́a-Cuetos et al., 2010). Kim et al. (2012) ont montré qu'après l'ingestion de Mesodinium, les plastes conservés chez D.caudata subissaient une certaine transformation et perdaient deux des quatre membranes qui les entourent. Ainsi, Dinophysis se reproduit en quelques heures, un processus qui a pu prendre des années d'évolution pour que les dinoflagellés hétérotrophes ancestraux acquièrent des plastes permanents.
\item A ce jour, Dinophysis représente le seul protiste kleptoplastidique montrant une division plastidiale en l'absence du noyau de la proie.
\end{itemize}

Article \cite{beedessee2020integrated} :
\begin{itemize}
    \item Les dinoflagellés présentent un intérêt particulier car ils présentent une diversité morphologique, une grande richesse en espèces et la capacité de survivre dans différentes niches écologiques.
    \item Ils contribuent également à la prolifération d'algues nuisibles (HAB), produisant souvent des toxines mortelles pour les organismes aquatiques et les humains. 
    \item L'état condensé persistant des chromosomes des dinoflagellés et leur organisation cristalline liquide, la perte de l'encapsidation de la chromatine nucléosomale, l'utilisation du 5-hydroxyméthyluracile dans l'ADN génomique nucléaire et les énormes génomes de certains dinoflagellés (≥ 100 Gbp) sont anormaux pour les eucaryotes. la compréhension de la biosynthèse des toxines des dinoflagellés reste insaisissable en raison de leurs génomes inhabituellement grands et des études de biosynthèse limitées. 
    \item Les algues toxique ont un squelette polyketide.
    \item La régulation de la biosynthèse des toxines a tendance à être coordonnée principalement au niveau transcriptionnel 
    \item Le squelette carboné de ces polycétides est généralement assemblé à partir d'acétate, avec le rare ajout de glycine pour former des polycétides hybrides (34). La glycine reste le seul substrat d'acides aminés signalé dans les métabolites isolés des dinoflagellés (35, 36), et notre analyse suggère que les affinités uniques pour le substrat du domaine d'adénylation NRPS contribuent à la complexité des métabolites chez les dinoflagellés. 
    \item Utilisation du séquencage de transcriptome profond pour analyser l'impacte du manque de nutriment sur la production des métabolites secondaires
    \item Régulation possible de la biosynthèse des toxines par les microARN en cas de carence en nutriments 

Article \cite{wisecaver2011dinoflagellate} :
\item Éléments utiles a l'introduction
\item Font partie du groupe Alveotala
\begin{itemize}
    \item Car identification de chromosome qui reste qui reste condensé durant le cycle cellulaire sans l'aide des nucléosomes 
    \item son génome fait partie des plus grands
    \item Possède une base inhabituelle : hydroxymethyluracil
\end{itemize}
\item dinokaryotique nuclei : organisation spécifique du noyau caractérisé par un chromosome condensé. 

\end{itemize}

Article \cite{wang2023transcriptome}
\begin{itemize}
\item Le profilage du transcriptome (methode utilisée pour étudier l'expression des gènes dans une cellule ou un groupe de cellules a un moment donné) révèle une réponse globale du dinoflagellé nocif Karlodinium veneficum aux algicides bactériens naturels
\item Les dinoflagellés nocifs peuvent produire des toxines qui s'accumulent tout au long de la chaîne alimentaire et provoquent des maladies ou la mort chez les animaux marins et les humains.
\item Shewanella sp. IRI-160 est une bactérie qui sécrète des composés algicides (IRI-160AA) ciblant spécifiquement les dinoflagellés.
\item Une analyse transcriptomique a été réalisée sur le dinoflagellé toxique Karlodinium veneficum exposé à l'IRI-160AA et à l'ammonium pour étudier les effets au niveau moléculaire.
\item Les résultats ont démontré des impacts différentiels de l'IRI-160AA et de l'ammonium sur K. veneficum, révélant une réponse globale de K. veneficum à l'exposition aux algicides.
\item Des gènes exprimés différemment associés au stress, aux dommages à l'ADN, à l'activation des points de contrôle du cycle cellulaire et à la mort cellulaire programmée ont été identifiés dans K. veneficum exposé à l'IRI-160AA.
\end{itemize}

Article \cite{bowazolo2023ribosome} : 
\begin{itemize}
    \item Les dinoflagellés montrent des changements dans la synthèse des protéines au cours du cycle quotidien
    \item  Les dinoflagellés sont des protistes unicellulaires abondants dans les écosystèmes marins.
    \item  Le profilage des ribosomes (technique utilisée pour étudier la traduction des ARNm en protéine dans les cellules)  a été utilisé pour mesurer les taux de traduction des transcriptions du dinoflagellé Fugacium kawagutii toutes les 2 heures sur un cycle clair-obscur.
    \item Plusieurs milliers de transcriptions montrent une augmentation des taux de traduction à des moments précis de la journée.
    \item  Les transcriptions réglementées ont été attribuées à différentes voies KEGG et les transcriptions qui ont été traduites à peu près au même moment ont été appelées simultanément réglementées.
    \item  Les résultats suggèrent que les cycles lumière-obscurité semblent capables de synchroniser la traduction de certaines transcriptions codant pour les protéines impliquées dans une gamme de différents processus cellulaires.
\end{itemize}

Article \cite{gaonkar2023novo }:
Possible transcriptome de référence 

Wikipedia : 

\textbf{Organisation unique de l'ADN :}

Le noyau des dinoflagellés, appelé "dinocaryon", présente une organisation de l'ADN très particulière :
\begin{itemize}
    \item Pendant l'interphase, l'ADN n'est pas associé à des histones comme chez les autres eucaryotes
    \item L'ADN est complexé à des protéines basiques spécifiques appelées DPVN (Dinoflagellate viral nucleoproteins)
    \item Ces DPVN proviennent probablement d'un transfert horizontal de gènes par des virus.
\end{itemize}

Malgré l'absence d'histones associées à l'ADN, les séquences codant pour les histones sont toujours présentes dans le génome des dinoflagellés et sont transcrites, bien que très divergentes par rapport aux autres eucaryotes
-> ARN interférence ?
-> Pourquoi on a toujours le gène ? peut etre en pleine évolution 

Expression des protéine régulé après la transcription par des microARN 

Article : \cite{place2012karlodinium}

Karlodinium veneficum : un petit dinoflagellé ayant des impacts écologiques significatifs. Il est associé à la mortalité de la faune aquatique dans le monde entier.
Il peut former des proliférations denses, atteignant des concentrations de cellules 104-105 mL1, conduisant souvent à la mort des poissons.
K. veneficum produit une série de composés toxiques appelés karlotoxines, qui ont des propriétés hémolytiques, ichtyotoxiques et cytotoxiques.
Ces toxines provoquent une dépolarisation de la membrane, une perturbation des fonctions motrices et une lyse cellulaire.
L'espèce a une répartition mondiale et a été signalée dans divers écosystèmes côtiers.

Article : \cite{park2019revisiting}

Dinophysis acuminata et D. ovum sont deux espèces similaires de dinoflagellés marins qui peuvent produire des proliférations d'algues nuisibles.
L'identification précise de ces espèces est difficile en raison de leurs similitudes morphologiques.
Une étude a été menée pour examiner les caractéristiques morphologiques et génétiques de 54 souches du "complexe D. acuminata" isolées dans les eaux côtières coréennes.
Les résultats suggèrent que D. acuminata et D. ovum peuvent être de la même espèce, avec des variations morphologiques influencées par des facteurs tels que la localisation géographique, les changements saisonniers et les conditions environnementales.
Le gène mitochondrial COX1 peut ne pas être un marqueur génétique approprié pour la discrimination entre ces deux espèces.

Article : \cite{stern2014diversity} 

Les cellules complexes de Dinophysis acuminata ont été collectées sur quatre sites côtiers en Écosse et analysées à l'aide de techniques moléculaires pour identifier les espèces et déterminer les types de plastes.
Un total de 32 isolats unicellulaires ont été obtenus, et le code-barres de l'ADN a été réalisé à l'aide de l'espaceur ribosomique transcrit interne 1 (ITS1) et d'un fragment partiel de cytochrome oxydase I (COI) .
La plupart des cellules ont été identifiées comme étant D. acuminata avec des plastes de type Teleaulax, qui est le type de plastide le plus fréquemment rapporté dans la dinophyse.
Cependant, une cellule a été identifiée comme étant D. cf. acuta et avait une plastide de type Storeatula/Rhodomonas, qui est le premier rapport de ce type de plastide dans les eaux écossaises.
L'étude met en évidence la diversité des espèces de Dinophysis et des types de plastes dans les eaux écossaises et fournit de nouvelles informations sur leurs cibles potentielles pour les proies.

https://www.sciencedirect.com/science/article/pii/S1568988311001491
\printbibliography % This prints the bibliography

Définitions
La stratification \textbf{thermohaline} désigne la séparation des couches d'eau dans l'océan en fonction de la température (thermo-) et de la salinité (-haline), deux facteurs qui influencent la densité de l'eau.

Le terme \textbf{holoplanctonique} désigne les organismes qui passent l'intégralité de leur cycle de vie dans le plancton, c’est-à-dire dans la colonne d’eau, flottant ou dérivant avec les courants. Ces organismes, appelés holoplancton, sont adaptés à la vie planctonique en permanence

Les métabolites secondaires sont des \textbf{composés phytochimiques non directement impliqués dans les processus vitaux de bases} (croissance, la division cellulaire, la respiration, la photosynthèse, reproduction).

Les enzymes de synthèse des peptides non ribosomiques, aussi appelées NRPS 

PKS : Une \textbf{polycétide synthase} est une enzyme multifonctionnelle ou un complexe enzymatique produisant des polycétides, une grande famille de métabolites secondaires chez les bactéries, les mycètes, les plantes et certaines lignées d'animaux 

Transcriptome : l'ensemble des transcrit d'ARN produit par le génome 
Profilage : vue globale de quels sont activés ou désactivés dans différentes conditions.

Trans-splicing of nuclear mRNA :
\end{document}


